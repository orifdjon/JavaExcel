\chapter{Компоненты Apache POI}
\label{chapter1}

%Большие отсупы --- это хорошо. Облегчает чтение длинных <<простыней>> текста
\section{Описание компонетов}

\begin{center}
	\begin{tabular}{ll}
	Horrible Spreadsheet Format & Компонент чтения и записи файлов MS-Excel, формат XLS \\
	XML Spreadsheet Format Компонент & Чтения и записи файлов MS-Excel, формат XLSX \\
	Horrible Property Set Format & Компонент получения наборов свойств файлов MS-Office \\
	Horrible Word Processor Format & Компонент чтения и записи файлов MS-Word, формат DOC \\
	XML Word Processor Format & Компонент чтения и записи файлов MS-Word, формат DOCX \\
	Horrible Slide Layout Format & Компонент чтения и записи файлов PowerPoint, формат PPT \\
	XML Slide Layout Format & Компонент чтения и записи файлов PowerPoint, формат PPTX \\
	Horrible DiaGram Format & Компонент работы с файлами MS-Visio, формат VSD \\
	XML DiaGram Format & Компонент работы с файлами MS-Visio, формат VSDX \\
	\end{tabular}
\end{center}



\section{Список компонентов}

\begin{table}[h!]
	\begin{center}
		\caption{Список компонентов\cite{OffDoc}}
		\label{tab:listOfComponents}
		\begin{tabular}{l|l}
			\textbf{Наименование(артефакт)} & \textbf{Необходимые компоненты} \\
			\hline
			poi & commons-logging, commons-codec, commons-collections, log4j \\
			poi-scratchpad & 	poi \\
			poi-ooxml & poi, poi-ooxml-schemas \\ 
			poi-ooxml-schemas & xmlbeans \\
			poi-examples & 	poi, poi-scratchpad, poi-ooxml \\
			ooxml-schemas & 	xmlbeans  \\
		    ooxml-security & xmlbeans \\
		\end{tabular}
	\end{center}
\end{table}

В данной работе рассматриваются следующие классы, используемые для работы с файлами Excel из приложений Java.
\begin{itemize}
	\item рабочая книга - HSSFWorkbook, XSSFWorkbook
	\item лист книги - HSSFSheet, XSSFSheet
	\item строка - HSSFRow, XSSFRow
	\item ячейка - HSSFCell, XSSFCell
	\item стиль - стили ячеек HSSFCellStyle, XSSFCellStyle
	\item шрифт - шрифт ячеек HSSFFont, XSSFFont
\end{itemize}

Поскольку описание всех классов и методов не разместить на одной странице, то можно ее прочитать в \href{https://poi.apache.org/apidocs/index.html}{офф док}.



