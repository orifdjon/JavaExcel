\lstset{frame=tb,
	language=Java,
	aboveskip=3mm,
	belowskip=3mm,
	showstringspaces=false,
	columns=flexible,
	basicstyle={\small\ttfamily},
	numbers=none,
	numberstyle=\tiny\color{gray},
	keywordstyle=\color{blue},
	commentstyle=\color{dkgreen},
	stringstyle=\color{mauve},
	breaklines=true,
	breakatwhitespace=true,
	tabsize=3
}

\definecolor{dkgreen}{rgb}{0,0.6,0}
\definecolor{gray}{rgb}{0.5,0.5,0.5}
\definecolor{mauve}{rgb}{0.58,0,0.82}


\chapter*{Введение}
\label{sec:afterwords}
\addcontentsline{toc}{chapter}{Введение}

\subsection*{Описание:}

В современном мире очень много случаев, при которых необходимо интегрировать MS
Excel с Java. Например, при разработке Enterprise-приложения в некой финансовой
сфере, вам необходимо предоставить счет для заинтересованных лиц, а проще всего
выставлять счет на MS Excel.

\subsection*{Обзор существующих API MS Excel для Java:}


Рассмотрим основные API\cite{APIMS}:
\begin{itemize}
	\item Docx4j - это  API с открытым исходным кодом, для создания и манипулирования документами формата Microsoft Open XML, к которым отросятся Word docx, Powerpoint pptx, Excel xlsx файлы. Он очень похож на Microsoft OpenXML SDK, но реализован на языке Java. Docx4j использует JAXB архитектуру для создания представления объекта в памяти. Docx4j акцентирует свое внимание на всесторонней поддержке заявленного формата, но от пользователя данного API требуется знание и понимание технологии JAXB и структуры Open XML.
	\item Apache POI - это набор API с открытым исходным кодом, который предлагает определенные функции для чтения и записи различных документов, базирующихся на Office Open XML стандартах (OOXML) и Microsoft OLE2 форматe документов (OLE2). OLE2 файлы включают большинство Microsoft Office форматов, таких как doc, xls, ppt. Office Open XML формат это новый стандарт базирующийся на XML разметке, и используется в файлах Microsoft office 2007 и старше.
	\item Aspose for Java - набор платных Java APIs, которые помогают разработчикам в работе с популярными форматами бизнес файлов, такими как документы Microsoft Word, таблицы Microsoft Excel, презентации Microsoft PowerPoint, PDF файлы Adobe Acrobat, emails, изображения, штрих-коды и оптические распознавания символов.

\end{itemize}
 

Каждое API проектируется для того, чтобы выполнять широкий спектр создания документов, различные манипуляции и преобразования быстро и легко, экономя время и позволяя разработчикам успешно программировать. Ни один API с открытым исходным кодом не имеет одной и той же комплексной поддержки функций.

Все Aspose’s APIs используют простую объектную модель документа, а одно API предназначено для работы с набором связанных форматов. Aspose’s Microsoft Office APIs, Aspose.Cells, Aspose.Words, Aspose.Slides, Aspose.Email, и Aspose.Tasks легки в работе, эффективны, надежны и независимы от других библиотек.

Преимуществом APIs с открытым исходным кодом является то, что они бесплатны и каждый может настроить их под свои задачи и цели. Это очень удобно, если у пользователя есть достаточно времени и ресурсов. Однако данные APIs не всегда имеют поддержку или документацию, и поддерживают небольшое количество функций и вариантов. Этот недостаток стоит разработчикам времени, и сокращает надежность их приложений. К преимуществам проприетарных (коммерческих) API можно отнести комплексную поддержку функционала с подробной документацией, регулярное обновление, гарантию отсутствия ошибок и обратную связь с разработчиками APIs.

В данной программе будем использовать Apache POI


