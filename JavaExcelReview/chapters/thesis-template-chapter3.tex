\lstset{frame=tb,
	language=Java,
	aboveskip=3mm,
	belowskip=3mm,
	showstringspaces=false,
	columns=flexible,
	basicstyle={\small\ttfamily},
	numbers=none,
	numberstyle=\tiny\color{gray},
	keywordstyle=\color{blue},
	commentstyle=\color{dkgreen},
	stringstyle=\color{mauve},
	breaklines=true,
	breakatwhitespace=true,
	tabsize=3
}

\definecolor{dkgreen}{rgb}{0,0.6,0}
\definecolor{gray}{rgb}{0.5,0.5,0.5}
\definecolor{mauve}{rgb}{0.58,0,0.82}


\chapter{Инструкция}

\begin{itemize}
	\item Для обращения к MS Excel версии до 2003 включительно года с Java используется класс HSSFWorkbook
	\item Для обращения к MS Excel версии 2007 и позднее с Java используется класс XSSFWorkbook
	\item  При операциях \textbf{Обновление} или \textbf{Запись} необходимо, чтобы MS Excel был закрыт.
\end{itemize}

\section{Чтение ячейки с MS Excel }

Чтобы считать данные с \textbf{file.xlsx} необходимо исполнить следующие шаги:

\begin{lstlisting}
	//filePath - это путь до MS Excel
	Workbook book = new XSSFWorkbook(new FileInputStream(filePath);
	//считывается лист по индексу sheet_index. sheet_index начинается с 0
	Sheet sheet = book.getSheetAt(sheet_index);
	//считывается row по индексу row_index. row_index начинается с 0
	Row row = sheet.getRow(row_index);
	//считывается cell по индексу cell_index. cell_index начинается с 0
	Cell cell = sheet.getCell(cell_index);
\end{lstlisting}

\section{Запись в ячейку MS Excel}

\begin{lstlisting}
	Workbook book = new XSSFWorkbook();
	//name - имя листа
	Sheet sheet = book.createSheet(name);
	Row row = sheet.createRow(i);
	Cell cell = row.createCell(j);
	FileInputStream fileOut = new FileInputStream(filePath);
	book.write(fileOut);
	fileOut.close();
\end{lstlisting}

\section{Обновление ячейки в существующем листе MS Excel}

\begin{lstlisting}
	Workbook workbook = new XSSFWorkbook(new FileInputStream(filePath));
	Sheet sheet = workbook.getSheetAt(i);
	Row row = sheet.getRow(j);
	Cell cell = row.getCell(k);
	cell.setCellValue(value);
\end{lstlisting}

\section{Подготовка: загрузка библиотек и зависимостей}

Чтобы использовать apache POI, вам нужно скачать jar файлы и добавить их через Intellij IDEA вручную, или вы можете предоставить это Maven.

Во втором случае вам нужно просто добавить следующие две зависимости в pom.xml:

\lstset{language=xml}

\begin{lstlisting}
	<dependencies>
		<dependency>
			<groupId>org.apache.poi</groupId>
			<artifactId>poi</artifactId>
			<version>3.12</version>
		</dependency>
		<dependency>
			<groupId>org.apache.poi</groupId>
			<artifactId>poi-ooxml</artifactId>
			<version>3.12</version>
		</dependency>
	</dependencies>
\end{lstlisting}

Самое удобное в Maven — что он загрузит не только указанные poi.jar и poi-ooxml.jar, но и все jar файлы, которые используются внутри, то есть xmlbeans-2.6.0.jar, 
stax-api-1.0.1.jar, poi-ooxml-schemas-3.12.jar и commons-codec-1.9.jar
