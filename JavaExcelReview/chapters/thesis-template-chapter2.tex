 \chapter{Классы и методы Apache POI для работы с файлами Excel}

\section{Рабочая книга HSSFWorkbook, XSSFWorkbook}

\begin{itemize}
	\item HSSFWorkbook org.apache.poi.hssf.usermodel класс чтения и записи файлов Microsoft Excel в формате .xls, совместим с версиями MS-Office 97-2003;
	\item XSSFWorkbook org.apache.poi.xssf.usermodel класс чтения и записи файлов Microsoft Excel в формате .xlsx, совместим с MS-Office 2007 или более поздней версии.
\end{itemize}

\subsection{Конструкторы класса HSSFWorkbook}

\begin{lstlisting}
	HSSFWorkbook ();
	HSSFWorkbook (InternalWorkbook book);
	HSSFWorkbook (POIFSFileSystem  fs);
	HSSFWorkbook (NPOIFSFileSystem fs);
	HSSFWorkbook (POIFSFileSystem  fs, boolean preserveNodes);
	HSSFWorkbook (DirectoryNode directory,
	POIFSFileSystem fs,
	boolean preserveNodes);
	HSSFWorkbook (DirectoryNode directory, boolean preserveNodes);
	HSSFWorkbook (InputStream s);
	HSSFWorkbook (InputStream s, boolean preserveNodes);
\end{lstlisting}

\textit{preservenodes является необязательным параметром, который определяет необходимость сохранения узлов типа макросы.}

\subsection{Конструкторы класса XSSFWorkbook}

\begin{lstlisting}
	XSSFWorkbook ();
	// workbookType  создать .xlsx или .xlsm
	XSSFWorkbook (XSSFWorkbookType workbookType);
	XSSFWorkbook (OPCPackage pkg );
	XSSFWorkbook (InputStream is);
	XSSFWorkbook (File file);
	XSSFWorkbook (String path);
\end{lstlisting}

\subsection{Основные методы HSSFWorkbook, XSSFWorkbook}

\begin{table}[h!]
	\begin{center}
		\caption{Основные методы HSSFWorkbook, XSSFWorkbook}
		\label{tab:listOfComponentsWithBooks}
		\begin{tabular}{l|l}
			\textbf{Метод} & \textbf{Описание} \\
			\hline
			createSheet() & Создание страницы книги HSSFSheet, XSSFSheet \\
			createSheet(String name) & Создание страницы с определенным наименованием \\
			createFont()  & Создание шрифта \\ 
			createCellStyle() & Создание стиля \\
		\end{tabular}
	\end{center}
\end{table}

С полным перечнем всех методов класса XSSFWorkbook можно познакомиться на странице \href{http://poi.apache.org/apidocs/org/apache/poi/xssf/usermodel/XSSFWorkbook.html}{http://poi.apache.org/apidocs/org/apache/poi/xssf/usermodel/XSSFWorkbook.html}.

\section{Классы листов книги, HSSFSheet, XSSFSheet}

\begin{itemize}
	\item org.apache.poi.hssf.usermodel.\textbf{HSSFSheet}
	\item org.apache.poi.xssf.usermodel.\textbf{XSSFSheet}
\end{itemize}

Классы HSSFSheet, XSSFSheet включают свойства и методы создания строк, определения размера колонок, слияния ячеек в одну область и т.д.

\subsection{Основные методы классов работы с листами}

\begin{table}[h!]
	\begin{center}
		\caption{Основные методы классов работы с листами}
		\label{tab:listOfComponentsWithList}
		\begin{tabular}{l|l}
			\textbf{Метод} & \textbf{Описание} \\
			\hline
			addMergedRegion(CellRangeAddress) & Определение области слияния ячеек страницы \\
			autoSizeColumn(int column) & Автоматическая настройка ширины колонки column (отсчет от 0) \\
			setColumnWidth(int column, int width)  & Настройка ширины колонки column (отсчет от 0) \\ 
			createRow(int row) & Создание строки row (отсчет от 0) \\
			getRow(int row) & Получение ссылки на строку row (отсчет от 0) \\
		\end{tabular}
	\end{center}
\end{table}

С полным перечнем всех методов класса XSSFSheet можно познакомиться на странице\\ \href{https://poi.apache.org/apidocs/org/apache/poi/xssf/usermodel/XSSFSheet.html}{https://poi.apache.org/apidocs/org/apache/poi/xssf/usermodel/XSSFSheet.html}.

\section{Классы строк HSSFRow, XSSFRow}

\begin{itemize}
	\item org.apache.poi.hssf.usermodel.\textbf{HSSFRow}
	\item  org.apache.poi.xssf.usermodel.\textbf{XSSFRow}
\end{itemize}

Классы HSSFRow, XSSFRow включают свойства и методы работы со строками, создания ячеек в строке и т.д.

\subsection{Основные методы классов HSSFRow, XSSFRow}

\begin{table}[h!]
	\begin{center}
		\caption{Основные методы классов HSSFRow, XSSFRow}
		\label{tab:listOfComponentsWithSheets}
		\begin{tabular}{l|l}
			\textbf{Метод} & \textbf{Описание} \\
			\hline
			setHeight(short) & Определение высоты строки \\
			getHeight() & Получение значения высоты в twips'ах (1/20) \\
			getHeightInPoints()  & Получение значение высоты \\ 
			createCell (int) & Создание ячейки в строке (отсчет от 0) \\
			getCell(int) & Получение ссылки на ячейку \\
			getFirstCellNum() & Получение номера первой ячейки в строке \\
			setRowStyle(CellStyle) & Определение стиля всей строки\\
		\end{tabular}
	\end{center}
\end{table}


С полным перечнем всех методов класса XSSFRow можно познакомиться на странице \href{http://poi.apache.org/apidocs/org/apache/poi/xssf/usermodel/XSSFRow.html}{http://poi.apache.org/apidocs/org/apache/poi/xssf/usermodel/XSSFRow.html}

\section{Классы ячеек HSSFCell, XSSFCell}

Ячейки электронной таблицы используются для размещения информации. В ячейке может быть представлено числовое значение, текст или формула. Также ячейка может содержать комментарий.

Классы HSSFCell, XSSFCell включают свойства и методы работы с ячейками таблицы.
\subsection{Основные методы классов HSSFCell, XSSFCell}


\begin{table}[!h]
	\begin{center}
		\caption{Основные методы классов HSSFRow, XSSFRow}
		\label{tab:listOfComponentsWithCell}
		\begin{tabular}{l|l}
			\textbf{Метод} & \textbf{Описание} \\
			\hline
			getBooleanCellValue() & Чтение логического значения ячейки \\
			getDateCellValue() & Чтение значения ячейки типа java.util.Date  \\
			getNumericCellValue()  & Чтение числового значения ячейки типа double \\ 
			getStringCellValue() & Чтение текстового значения ячейки (java.lang.String) \\
			setCellValue(boolean) & Опреaделение логического значения ячейки \\
			setCellValue(java.util.Calendar) & Определение значения ячейки типа даты \\
			setCellValue(java.util.Date) & Определение значения ячейки типа даты\\
			getCellTypeEnum() & Чтение типа значения ячейки CellType \\
			setCellComment(Comment) & Запись комментария в ячейку \\
			getCellComment() & Чтение комментария ячейки \\
			removeCellComment() & Удаление комментария ячейки \\
			setHyperlink(Hyperlink) & Запись гиперссылки в ячейку \\
			getHyperlink() & Чтение гиперссылки XSSFHyperlink в ячейке \\
			removeHyperlink() & Удаления гиперссылки ячейки \\
			getCellFormula() &  Чтение формулы, например SUM(C4:E4) \\
			setCellFormula(String) & Определение формулы, например =SUM(C4:E4) \\
			getCellStyle() & Чтение стиля ячейки (XSSFCellStyle) \\
			setCellStyle(CellStyle) & Определение стиля ячейки \\
			getColumnIndex() & Определение индекса ячейки \\
			setAsActiveCell() & Определение активности ячейки \\
		\end{tabular}
	\end{center}
\end{table}

-   org.apache.poi.hssf.usermodel.\textbf{HSSFCell}
-   org.apache.poi.xssf.usermodel.\textbf{XSSFCell}
\\
С полным перечнем всех методов класса XSSFCell можно познакомиться на странице \\ \href{http://poi.apache.org/apidocs/org/apache/poi/xssf/usermodel/XSSFCell.html}{http://poi.apache.org/apidocs/org/apache/poi/xssf/usermodel/XSSFCell.html}



\section{Классы стилей ячеек HSSFCellStyle, XSSFCellStyle}

С полным перечнем всех свойств и методов класса XSSFCellStyle можно познакомиться на странице \\
\href{http://poi.apache.org/apidocs/org/apache/poi/ss/usermodel/CellStyle.html}{http://poi.apache.org/apidocs/org/apache/poi/ss/usermodel/CellStyle.html} \\
Ниже в качестве примера представлен метод, формирующий стиль ячейки, в которой :

-   текст центрируется по вертикали и горизонтали;
-   обрамление ячейки представляет тонкую черную линию по периметру;
-   текст переносится на следующую строку (не ячейку), если не вмещается в размер ячейки.

\begin{lstlisting}
private XSSFCellStyle createCellStyle(XSSFWorkbook book) {
	BorderStyle thin  = BorderStyle.THIN;
	short       black = IndexedColors.BLACK.getIndex();

	XSSFCellStyle style = book.createCellStyle();

	style.setWrapText(true);
	style.setAlignment        (HorizontalAlignment.CENTER);
	style.setVerticalAlignment(VerticalAlignment  .CENTER);
	
	style.setBorderTop        (thin);
	style.setBorderBottom     (thin);
	style.setBorderRight      (thin);
	style.setBorderLeft       (thin);
	
	style.setTopBorderColor   (black);
	style.setRightBorderColor (black);
	style.setBottomBorderColor(black);
	style.setLeftBorderColor  (black);
	
	return style;
}
\end{lstlisting}

Метод setWrapText позволяет определить флаг переноса текста в ячейке согласно ее размеру (ширине). Чтобы перенести текст принудительно, можно в текстовой строке установить символы CRCL, например "Разделитель $\backslash$r $\backslash$n текста".

\section{Классы шрифтов HSSFFont, XSSFFont}

С полным перечнем всех свойств и методов класса XSSFFont можно познакомиться на странице\\
\href{http://poi.apache.org/apidocs/org/apache/poi/ss/usermodel/Font.html}{http://poi.apache.org/apidocs/org/apache/poi/ss/usermodel/Font.html}

Ниже в качестве примера представлен метод, формирующий шрифт типа "Times New Roman" :

\begin{lstlisting}
private XSSFFont createCellFont(XSSFWorkbook book)  {  	
	XSSFFont font = workBook.createFont(); 
	font.setFontHeightInPoints((short)  12); 
	font.setBoldweight(XSSFFont.BOLDWEIGHT_BOLD); 
	font.setFontName("Times New Roman");  
	return(font);  
}  
.  .  .  
HSSFCellStyle style = book.createCellStyle(); style.setFont(createCellFont(book));
\end{lstlisting}
